\chapter{Introduction}
\minitoc
The study of impurity transport is related to the fuel dilution. \\
ITER will feature tungsten (W) tiles in the divertor and highly radiating elements could be additionally injected to build up a radiation shiled at the edge to protect the wall \cite{Valisa}. \\
From the paper of Valisa \cite{Valisa}
\begin{itemize}
    \item p2) additional central heating and especially electron heating has a flattening effect on the density profiles of impurities in the plasma core and may reverse the impurity radial convection from begin typically inwards to outwards
    \item   az
\end{itemize}


The classical inward diffusion of high-Z impurities in toroidal plasmas is enhanced by the Pfirsch-Schlüter effect. It is found that both density and temperature gradients produce inward impurity diffusion \cite{Rutherford}. In the limit where the mass of the impurity ion is very large compared with the mass of the proton, the proton-impurity collisions may be treated by a \emph{Lorentz model} / {\bf{Lorentz model}}. In this limit, the problem of transport of protons due to proton-impurity collisions is similar to the problem of transport of electrons due to electron-ion collisions. \\

The neoclassical theory of plasma transport in axisymmetric is developed by means of a variationial principle for the rate of irreversible entropy production in the paper of \emph{Rosenbluth} \cite{Rosenbluth}. \\

The neoclassical theory predicts important phenomena in tokamaks such as the bootstrap current, electric conductivity, transport in the scrape-off layer, and cross-field transport in regions where the turbulence is suppressed. \\

It is widely recognized that the conventional theory of neoclassical transport in tokamaks is not applicable to regions where the pressure and temperature profiles are very steep, such as the pedestal at the plasma edge. The reason for this lies in the orderings of the theory. 


\iffalse
\begin{figure}[h]
\begin{center}
	\subfloat{\includegraphics[width=8cm]{density_spec1.png}}\label{fig:1aa} 
	\subfloat{\includegraphics[width=8cm]{density_spec2.png}}\label{fig:1ab} \\
	\subfloat{\includegraphics[width=8cm]{density_spec4.png}}\label{fig:1ac} 
	\subfloat{\includegraphics[width=8cm]{density_spec3.png}}\label{fig:1ad} 
    \caption{Density profile $N_z$ with different initial value $\kappa_{nz}=\pm 0.3, \pm 0.5$. (Solid) initial density profile at $t=0$. (Dashed) density profile at $t=2$}
    \label{density_gradient}
\end{center}
\end{figure}
\fi