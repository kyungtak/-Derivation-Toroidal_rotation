\chapter{The effect of toroidal rotation on impurity transport}
\minitoc
In general, neoclassical transport in tokamak is establisehd in the limit where the ion thermal speed is much larger than the toroidal rotation speed of the plasma. However, large toroidal rotation is possible especially during the neutral beam injection. \\
In this section, we will demonstrate how toroidal rotation in tokamak can modify impurity transport. Usually, strong toroidal rotation has been found to increase neoclassical transport. The analytical derivation will be given below to prove this effect. The following derivation is strongly inspired by \citep{Hinton85, Helander07, Catto87}.
\begin{equation}
    u_\parallel \nabla f_{a1} - C_a(f_{a1}) = -u_\parallel \overbar{f}_{a0}\sum_{j=1}^3 A_j \nabla_\parallel \alpha_j
\end{equation}
where $f_{a1}=\overbar{f}_a-\overbar{f}_{a0}=\overbar{f}_{a*}+\overbar{g}_a-\overbar{f}_{a0}$ and the thermodynamic forces are defined
\begin{align}
    A_1 &= \frac{d\ln{N_a}}{d\psi} + \frac{d\ln{T_a}}{d\psi} \\
    A_2 &= \frac{d\ln{T_a}}{d\psi} \\
    A_3 &= \frac{d\ln{\omega}}{d\psi}
\end{align}
and we have written
\begin{align}
    \alpha_1 &=\frac{m_a R \overbar{v}_\varphi}{e_a} = \frac{m_a}{e_a}\Big(\omega R^2 + \frac{I u_\parallel}{B}\Big) \\
    \alpha_2 &= \big (\frac{H}{T_a}-\frac{5}{2}\Big)\alpha_1 \\
    \alpha_3 &= \frac{m_a^2 \omega R^2 \overbar{v}_\varphi^2}{2e_aT_a} = \frac{m_a^2 \omega}{2 e_a T_a}\Bigg[\Big(\omega R^2 + \frac{I u_\parallel}{B}\Big) + \frac{\mu R^2 B_p^2}{m_a B}\Bigg]
\end{align}
Therefore, the rotation shaer $A_3=d\ln{\omega}/d\psi$ enters naturally as a thrid driving force for transport in addition to the pressure and temperature gradients.

\section{Step 1}
The kinetic energy in the rotating frame is equal to
\begin{align}
    \frac{m_a u^2}{2} &= \frac{m_a(\bm{v}-\bm{V}_0)^2}{2} = \frac{m_a (\bm{v}^2+\bm{V}_0^2)}{2}-m_a \bm{v} \cdot \bm{V}_0 \nonumber \\ 
    &= \frac{m_a(v^2+R^2 \omega^2)}{2}-m_a v_\varphi R \omega \nonumber \\
    &= \frac{m_a(v^2+R^2 \omega^2)}{2} + mRv_\varphi \frac{d\phi_{-1}}{d\psi} \nonumber \\
    &= \boxed{\frac{m_a}{2}(v^2+R^2\omega^2) + e_a  \frac{d\phi_{-1}}{d\psi}(\psi-\psi^*)} \label{rotating_kinetic_energy}
\end{align}
where we used the relations 
\begin{align}
    \bm{V}_0 &= \omega(\psi)R \hat{\varphi} = -R \frac{d\phi_{-1}}{d\psi} \hat{\varphi} \label{flow} \\
    \psi^* &= \psi - m_a R v_\varphi / e_a = -p_\varphi / e_a \label{cnst_motion}
\end{align}
where Eq.(\ref{cnst_motion}) is a constant of motion due to axysimmetry.\\
Inserting Eq.(\ref{rotating_kinetic_energy}) into the Maxwellian distribution function in a rotating frame 
\begin{align}
    f_{a0}&=n_a \Big( \frac{m_a}{2\pi T_a(\psi)} \Big)^{3/2} \exp{-\frac{m_a u^2}{2T_a(\psi)}} \nonumber \\
    &= n_a \Big( \frac{m_a}{2\pi T_a(\psi)} \Big)^{3/2} \exp{-\frac{m_a}{2T_a}(v^2+R^2\omega^2)-\frac{e_a}{T_a}\frac{d\phi_{-1}}{d\psi}(\psi-\psi^*)} \nonumber \\
    &\text{Then, insert $n_a=N_a(\psi,\theta)\exp{\frac{m_a \omega^2 R^2}{2T_a}-\frac{e_a\phi_0}{T_a}}$} \nonumber \\
    &= N_a(\psi) \Big( \frac{m_a}{2\pi T_a(\psi)} \Big)^{3/2} \exp{-\frac{m_a v^2}{2T_a}-\frac{e_a}{T_a}\frac{d\phi_{-1}}{d\psi}(\psi-\psi^*)-\frac{e_a \phi_0}{T_a}} \nonumber \\
    &=N_a(\psi) \Big( \frac{m_a}{2\pi T_a(\psi)} \Big)^{3/2} \exp{-\frac{H}{T_a}}
\end{align}
where 
\begin{align}
    H = \frac{m_a v^2}{2}+e_a\frac{d\phi_{-1}}{d\psi}(\psi-\psi^*)+e_a \phi_0
\end{align}

The function $H$ we defined is almost, but not quite, a constant of motion to the requisite accuracy. A true constant of motion is
\begin{align}
    H_* = \frac{m_a v^2}{2}+e_a[\phi(\psi)-\phi_{-1}(\psi)] \simeq H + e_a \frac{\partial^2 \phi_{-1}}{\partial \psi^2}\frac{(\psi-\psi_*)^2}{2}
\end{align}
We now introduce a function which is almost equal to $f_{a0}$ but only depend on constatns of motion.
\begin{align}
    f_{a*} = N_a(\psi_*)\Big(\frac{m_a}{2\pi T_a(\psi_*)}\Big)^{3/2}\exp{-\frac{H_*}{T_a(\psi_*)}}
\end{align}
Then, we write the distribution functino as $f_a = f_{a*}+g_a$ where $g_a \ll f_{a*}$ so that the Fokker-Planck equation becomes
\begin{align}
    \frac{df_a}{dt} &= \frac{\partial f_a}{\partial t} + \dot{z}_k \frac{\partial f_a}{\partial z_k} = \mathcal{C}_a(f_a - f_{a0}) \quad \text{(General form)} \\
    \frac{d g_a}{dt} &= \mathcal{C}_a(f_{a*}-f_{a0}+g_a) 
\end{align}
The drift kinetic equation follows upon gyroaveraging
\begin{align}
    v_\parallel \nabla_\parallel \overbar{g}_a = \overbar{\mathcal{C}_a(f_{a*}-f_{a0}+g_a)} = \mathcal{C}_a(\overbar{f_{a*}-f_{a0}+g_a})
\label{DK}
\end{align}
where we have neglected the corss-field drift on the left since $g_a$ is small and the time derivative has been neglected according to the transport ordering \footnote{$\partial / \partial t \sim D/L^2 \sim \delta^2\nu$ where $\delta=\rho/L \ll 1$} and we have used \hl{the rotational invariance of the collision operator}. \\
Using the facts, 
\begin{align}
    \frac{f_{a*}-f_{a0}}{\psi_* - \psi} &\simeq \frac{\partial f_{a0}}{\partial \psi} \\
    \frac{H_*-H}{\psi_*-\psi} &\simeq \frac{\partial H}{\partial \psi}
\end{align}
The difference between $f_{a*}$ and $f_{a0}$ can be written as 
\begin{align}
    f_{a*}-f_{a0} &\simeq (\psi_* - \psi)\Bigg[\frac{d\ln{N_a}}{d\psi}+\Big(\frac{H}{T_a}-\frac{3}{2}\Big)\frac{d\ln{T_a}}{d\psi}-\frac{1}{T_a}\frac{\partial H}{\partial \psi}\Bigg]f_{a0} \\
    &\simeq (\psi_* - \psi)\Bigg[\frac{d\ln{N_a}}{d\psi}+\Big(\frac{H}{T_a}-\frac{3}{2}\Big)\frac{d\ln{T_a}}{d\psi}\Bigg]f_{a0}-\frac{H_*-H}{T_a}f_{a0} \\
    &=-\frac{m_a R v_\varphi}{e_a}\Bigg[\frac{d\ln{N_a}}{d\psi} + \Big(\frac{H}{T_a}-\frac{3}{2}\Big)\frac{d\ln{T_a}}{d\psi} + \frac{m_a R v_\varphi}{2T_a}\frac{d\omega}{d\psi}\Bigg] f_{a0}
\end{align}
where we used the relations Eq.(\ref{flow}, \ref{cnst_motion}) to derive final form. \\
Also, using $\bm{v}=\omega R \hat{\varphi} + \bm{u}$ with $I=RB_\varphi$, the gyro-average of plasma flow gives 
\begin{align}
    \overbar{v_\varphi} &= \frac{I u_\parallel}{RB} + \omega R \nonumber \\
    \overbar{v_\varphi^2} &= \overbar{(v_\parallel \cdot \hat{\varphi})^2 + (v_\perp \cdot \hat{\varphi})^2} = \Big(\frac{I u_\parallel}{RB}+\omega R \Big)^2 + \frac{v_\perp^2}{2}\frac{B_p^2}{B_2}
\end{align}
Then we can finally write Eq.(\ref{DK}) as 
\begin{equation}
    u_\parallel \nabla f_{a1} - C_a(f_{a1}) = -u_\parallel \overbar{f}_{a0}\sum_{j=1}^3 A_j \nabla_\parallel \alpha_j
\end{equation}
where $f_{a1}=\overbar{f}_a-\overbar{f}_{a0}=\overbar{f}_{a*}+\overbar{g}_a-\overbar{f}_{a0}$ and the thermodynamic forces are defined
\begin{align}
    A_1 &= \frac{d\ln{N_a}}{d\psi} + \frac{d\ln{T_a}}{d\psi} \\
    A_2 &= \frac{d\ln{T_a}}{d\psi} \\
    A_3 &= \frac{d\ln{\omega}}{d\psi}
\end{align}
and we have written
\begin{align}
    \alpha_1 &=\frac{m_a R \overbar{v}_\varphi}{e_a} = \frac{m_a}{e_a}\Big(\omega R^2 + \frac{I u_\parallel}{B}\Big) \\
    \alpha_2 &= \big (\frac{H}{T_a}-\frac{5}{2}\Big)\alpha_1 \\
    \alpha_3 &= \frac{m_a^2 \omega R^2 \overbar{v}_\varphi^2}{2e_aT_a} = \frac{m_a^2 \omega}{2 e_a T_a}\Bigg[\Big(\omega R^2 + \frac{I u_\parallel}{B}\Big) + \frac{\mu R^2 B_p^2}{m_a B}\Bigg]
\end{align}
Therefore, the rotation shaer $A_3=d\ln{\omega}/d\psi$ enters naturally as a thrid driving force for transport in addition to the pressure and temperature gradients.

\section{From Wong's paper}
The following derivation comes from the seminal paper of \citep{Wong87}. \\
The neoclassical theory of transport is based on a multiple time scale expansion of the Fokker-Planck equation in the small parameter of gyroradius over plasma scale length. When the toroidal rotation becomes fast enough, \hl{the self-consistent electric field is considered to be larger by one order.} In Hinton's work from \citep{Hinton85}, the derivation of the drift kinetic equation and the expressions of the fluxes was detailed but only in case of pure plasma. Therefore, the formulation is extended to include more than on ion species. \\
In his work, the Fokker-Planck equations are expanded, leading to the linearized drift kinetic equations, then the flux averaged transport equations are derived and the expressions for the neoclassical fluxes obtained. 
\subsection{Expansion of Fokker-Planck equation}
The neoclassical theory of transport is based on a multiple time scale expansion of FP equation in the small parameter. The strong rotation theory can make difference of \hl{the self consistent electric field which need to be considered to be larger by one order.}\\
The Fokker-Planck equation for the ion distribution function is 
\begin{align}
    \frac{\partial f_i}{\partial t}+\bm{v} \cdot \nabla f_i + \frac{e}{m_i}\Big( \bm{E} + \frac{\bm{v}}{c}\times \bm{B}\Big) \cdot \frac{\partial f}{\partial \bm{v}} = \mathcal{C} + \mathcal{S}
\end{align}
where $\mathcal{C}$ is the collision operator and $\mathcal{S}$ is a source term, representing the effects of neutral
beam injection. \\
We transform to a local reference frame, moving with the velocity $\bm{u}_0$ relative to the lab frame, where the velocity is 
\begin{align}
    \bm{v}' = \bm{v}-\bm{u}_0(\bm{x},t)
\end{align}
Using the derivative relation
\begin{align}
    \frac{\partial f}{\partial t} &= \frac{\partial f}{\partial \bm{v}'} \frac{\partial \bm{v}'}{\partial t} = \frac{\partial f}{\partial \bm{v}'}\Big(\frac{\partial \bm{v}}{\partial t} - \frac{\partial \bm{u}_0}{\partial t}\Big) \nonumber \\
    \nabla f &= \frac{\partial f}{\partial \bm{v}'}\frac{\partial \bm{v}'}{\partial x} = \frac{\partial f}{\partial \bm{v}'}(\nabla \bm{v} - \nabla \bm{v'}) \nonumber \\
    \frac{\partial f}{\partial \bm{v}} &= \frac{\partial f}{\partial \bm{v}'}\frac{\partial \bm{v}'}{\partial \bm{v}}=\frac{\partial f}{\partial \bm{v}'}
\end{align}
Then, the transformed FP equation becomes : 
\begin{align}
    \frac{\partial f}{\partial t} + (\bm{v}' + \bm{u}_0) \cdot \nabla 'f + \frac{e}{m}\Big( \bm{E} + (\bm{v}'+\bm{u}_0) \times \frac{\bm{B}}{c}\Big)\cdot \frac{\partial f}{\partial \bm{v}'} \nonumber \\
    -\Big(\frac{\partial \bm{u}_0}{\partial t}+ (\bm{v}'+\bm{u}_0)\cdot (\nabla \bm{u}_0)\Big) \cdot \frac{\partial f}{\partial \bm{v}'}= \mathcal{C} + \mathcal{S}
\end{align}
where we used the following relations : 
\begin{align}
    \frac{\partial f}{\partial \bm{v}'} \nabla \bm{v} &= \nabla 'f \qquad \text{(why?)} \\
    \frac{\partial f}{\partial \bm{v}'} \frac{\partial \bm{v}}{\partial t} &= \frac{\partial f}{\partial t} \qquad \text{(why?)}
\end{align}
and here $\nabla '$ denotes the spatial gradient holding $\bm{v}'$ fixed. \\
Now, the distribution function is expanded
\begin{align}
    f = f_0 + f_1 + f_2 + \cdots
\end{align}
where the subscripts indicates the order in $\epsilon \sim \nu_i / \Omega_i$ and the time dependence is assumed to occur on well separated time scales, so that the time derivative may be formally expanded
\begin{align}
    \frac{\partial}{\partial t} = \frac{\partial}{\partial t_0} + \frac{\partial}{\partial t_1} + \frac{\partial}{\partial t_2} + \cdots
\end{align}
with the different terms of order $\bm{v_i}/L, \epsilon \bm{v_i}/L , \epsilon^2 \bm{v_i}/L$.
\subsection{Transport across magnetic surfaces}
The transport across magnetic surfaces must occur on the $t_2$ time scale, with $\partial / \partial t \sim \epsilon^2 \rho_i / L \sim \nu_{ii}(\rho_i / L)^2$.\\
The fluxes that are needed to determine the changes in density, temperature, and toroidal angular velocity caused by radial transport are the particle flux, energy flux, and momentum flux to second order in ion gyroradius over gradient  length. \\
The second order flux
\begin{align}
    \Gamma_2 &= -\frac{c}{e}\Big\langle \int d^3\bm{v} mRv_\varphi \mathcal{C}^1 f_1 \Big\rangle \\
    \Pi_2 &= -\frac{mc}{e}\Big\langle \int d^3\bm{v}\frac{1}{2}mR^2v_\varphi^2\mathcal{C}^1f_1\Big\rangle \\
    Q_2 &= -\omega \Pi_2 -\frac{mc}{e}\Big\langle \int d^3\bm{v}Rv_\varphi \big(\frac{1}{2}mv^2+e\Tilde{\phi}_0\big)\mathcal{C}^1f_1\Big\rangle
\end{align}
Here, we define the second order heat flux as \begin{align}
    q_2 = Q_2 -\omega \Pi_2 -\frac{5}{2}T\Gamma_2
\end{align}
These fluxes may be expressed explicitly as sums of neoclassical and classical parts by using $f_1 = \overbar{f_1}+\widetilde{f_1}$ and expressing the integrands in terms of $v_\parallel, v_\perp$ and $\xi$ and making the choice $\hat{e_1}=\nabla \Psi / \abs{\nabla \Psi}$. Then, we obtain
\begin{align}
    \Gamma_2 &=\overbar{\Gamma}_2 - \frac{mc}{e}\Big \langle \int d^3\bm{v}R(\hat{e}_\varphi \cdot \hat{e}_2)v_\perp \sin{\xi} \mathcal{C}^1 f_1 \Big\rangle \\
    q_2&=\overbar{q}_2 - \frac{mc}{e}\Big\langle \int d^3\bm{v} R (\hat{e}_\varphi \cdot \hat{e}_2)v_\perp \sin{\xi}\big(mE-\frac{5}{2}T\big)\mathcal{C}^1f_1\Big\rangle \\
    \Pi_2 &= \overbar{\Pi}_2 - \frac{mc}{e}\Big\langle \int d^3\bm{v}\frac{m}{2}R^2[2(\omega R + v_\parallel \hat{\bm{b}}\cdot \hat{e}_\varphi)v_\perp(\hat{e_2}\cdot \hat{e}_\varphi) \times \sin{\xi} -(v_\perp^2/2)(\hat{e}_2\cdot \hat{e}_\varphi)^2 \cos{2\xi}]\mathcal{C}^1f_1\Big\rangle
\end{align}
where the neoclassical part 
\begin{align}
    \overbar{\Gamma_2} &= -\Big\langle \int d^3\bm{v} \alpha_1 \mathcal{C}^1 \overbar{f}_1 \Big\rangle \\
    \overbar{q}_2 &= -\Big\langle \int d^3\bm{v}T\alpha_2 \mathcal{C}^1 \overbar{f}_1 \Big\rangle \\
    \overbar{\Pi}_2 &= -\Big\langle \int d^3\bm{v} \frac{T}{\omega}\alpha_3 \mathcal{C}^1 \overbar{f}_1 \Big\rangle
\end{align}
where the quantities $\alpha_1, \alpha_2$ and $\alpha_3$ defined by 
\begin{align}
    \alpha_1 &= (mc/e)[I(v_\parallel /B) + \omega R^2] \\
    \alpha_2 &= (mE/T - 5/2)\alpha_1 \\
    \alpha_3 &= \frac{mc\omega}{ev_i^2}\Big[\Big(\frac{Iv_\parallel}{B}+\omega R^2\Big)^2 + \mu \frac{\abs{\nabla \Psi}^2}{B}\Big]
\end{align}
The values of alpha defined here are equal to those described in Helander's book. Therefore, deriving Wong's paper in detail may help me to understand the neoclassical transport influenced by toroidal rotation and its possible effect.